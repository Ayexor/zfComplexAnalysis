% !TeX spellcheck = en_GB
% LaTeX header for lecture notes and exercises
\documentclass[11pt,a4paper]{article}
\usepackage[utf8]{inputenc}
\usepackage[T1]{fontenc}
\usepackage{amsmath}
\usepackage{amsfonts}
\usepackage{amssymb}
\usepackage{amsthm}
\usepackage{csquotes}
\usepackage{array}
\usepackage{enumerate}
\usepackage[english]{babel}
\usepackage[left=2cm,right=2cm,top=2cm,bottom=2cm]{geometry}
\usepackage{graphicx}
\usepackage{mathtools}
\usepackage{siunitx}
\usepackage{tensor}
\usepackage{txfonts}
\usepackage{tikz}
\usepackage{pgfplots}
\usepackage{algorithm2e}

\usepackage[
  colorlinks=false,% hyperlinks will be black
  urlbordercolor=blue,% hyperlink borders will be red
  pdfborderstyle={/S/U/W 1}% border style will be underline of width 1pt
  ]{hyperref}

% Ensure the document compiles even after updates.
% PGFplots does not alwasy guarantee backwards compatibility
\pgfplotsset{compat=1.17}

% Define numbering for these environments based on the section counter
% (Equal to predefined 'theorem' environment)
\theoremstyle{definition}
\newcounter{myDefCounter}[section]
\newtheorem{definition}[myDefCounter]{Definition}
\numberwithin{myDefCounter}{section}
\newtheorem{lemma}[myDefCounter]{Lemma}
\newtheorem{proposition}[myDefCounter]{Proposition}
\newtheorem{theorem}[myDefCounter]{Theorem}
\newtheorem{corolary}[myDefCounter]{Korollar}
\newtheorem{example}[myDefCounter]{Beispiel}
\newtheorem{task}[myDefCounter]{Aufgabe}

% Use a circle instead of a square to end proofs. Looks so much nicer...
\renewcommand{\qedsymbol}{$\bigcirc$}

% Simple 2x2 matrix
\newcommand{\mat}[4]{\begin{pmatrix*}[r] #1 & #2 \\ #3 & #4 \end{pmatrix*}}
\newcommand{\matc}[4]{\begin{pmatrix} #1 & #2 \\ #3 & #4 \end{pmatrix}}

% Use \rep{a}{b}{c} for the representation matrix of homomorphism b : <c> \to <a>
\newcommand{\rep}[3]{\tensor[_{#1}]{\lbrack #2 \rbrack}{_{#3}}}
% Span operator: \spn{B} creates <B>
\newcommand{\spn}[1]{\ensuremath{\operatorname{\left\langle #1 \right\rangle}}}
% Trace operator
\DeclareMathOperator*{\trace}{Tr}
% Image operator
\newcommand{\img}[1]{\ensuremath{\operatorname{im} #1}}
% Rank operator
\newcommand{\rank}[1]{\ensuremath{\operatorname{rank} #1}}
% arg min and arg max operators
\DeclareMathOperator*{\argmin}{arg\,min}
\DeclareMathOperator*{\argmax}{arg\,max}
% Polynomial degree operator (\deg already used for degree symbol (like in \deg C[elsius]))
\DeclareMathOperator{\grad}{\operatorname{grad}}
% Volume differential
\DeclareMathOperator{\dvol}{\operatorname{dvol}}
% ggT
\DeclareMathOperator{\ggT}{\operatorname{ggT}}

% Table column types for left / right / centered allignment with fixed width, use like: R{3em}
\newcommand{\PreserveBackslash}[1]{\let\temp=\\#1\let\\=\temp}
\newcolumntype{C}[1]{>{\PreserveBackslash\centering}p{#1}}
\newcolumntype{R}[1]{>{\PreserveBackslash\raggedleft}p{#1}}
\newcolumntype{L}[1]{>{\PreserveBackslash\raggedright}p{#1}}

%% Listings:
\usepackage{listings}
\usepackage{color} %red, green, blue, yellow, cyan, magenta, black, white
\definecolor{mygreen}{RGB}{28,172,0} % color values Red, Green, Blue
\definecolor{mylilas}{RGB}{170,55,241}


\lstset{
  extendedchars=true,
  basicstyle=\ttfamily,
  literate=%
  {€}{\euro}1%
  {§}{\S}1%
  {°}{\textdegree{}}1%
  {ä}{{\"a}}1%
  {ö}{{\"o}}1%
  {ü}{{\"u}}1%
  {ß}{{\ss}}1%
  {Ä}{{\"A}}1%
  {Ö}{{\"O}}1%
  {Ü}{{\"U}}1%
  {µ}{\textmu}1%
  {¹}{{\textsuperscript{1}}}1%
  {²}{{\textsuperscript{2}}}1%
  {³}{{\textsuperscript{3}}}1%
  {¼}{\textonequarter}1%
  {½}{\textonehalf}1%
  {¢}{\textcent}1%
}

\lstdefinestyle{Matlab}{
  numbers=left,
  belowcaptionskip=1\baselineskip,
  breaklines=true,
  frame=L,
  linewidth=13cm,
  xleftmargin=1cm,
  language=Matlab,
  showstringspaces=false,
  basicstyle=\footnotesize\ttfamily,
  keywordstyle=\bfseries\color{green!40!black},
  commentstyle=\itshape\color{purple!40!black},
  identifierstyle=\color{blue},
  stringstyle=\color{orange},
  numberstyle=\ttfamily\tiny,
  morestring=*[d]{"},
  numbersep=9pt,
  captionpos=b
}
\lstdefinelanguage{Matlab}{
  keywords={function,global,zeros,switch,case,otherwise,end,sin,cos,cot,%
    floor,ode45,hold,polarplot,for,if,else,norm,abs,diag,sqrt,randn,hess,format,short,long},
  sensitive=true
}


\author{Roy Seitz}
\title{Complex Analysis\\Summary}
\begin{document}
\maketitle
\tableofcontents

\section{Complex numbers and the complex plane}
\begin{definition}
  We denote the \textbf{open} and \textbf{closed disks} of radius $r>0$ around $z_0 \in \mathbb C$ with
  \[ D_r(z_0) \coloneqq \{z \in \mathbb z \mid |z - z_0| < r\}
  \qquad\text{and}\qquad
  \overline D_r(z_0) \coloneqq \{z \in \mathbb z \mid |z - z_0| \le r\}.\]
  We call $z \in \Omega$ an \textbf{interior point} if there exists an $r > 0$ such that $D_r(z) \subset \Omega$.
\end{definition}
And more stuff like continuity, open covers, compact sets\textellipsis

\section{Functions on the complex plane}
Let $\Omega \subset \mathbb C$ be a set,
$U \subset \mathbb C$ be open
and $f\colon U \to \mathbb C$ a function.
We denote the real and imaginary part as
$z = x + iy$ and $f = u + iv$.

\begin{definition}
  Letz $z_0 \in U$ and $h \in \mathbb C$.
  $f$ is called \textbf{holomorphic in $z_0$} if
  \[f'(z_0) = \lim_{h \to 0} \frac{f(z_0 + h) - f(z_0)}{h}\]
  exists.
  If $f$ is holomorphic for all $z \in U$, $f$ is called \textbf{holomorphic (on $U$)}.
  If $f$ is defined and holomorphic for all $z \in \mathbb C$, $f$ is called an \textbf{entire function}.
  The set of all holomorphic functions on $U$ is denoted
  \[\mathcal H(U) \coloneqq \{f\in \prescript{U}{}{\mathbb C} \mid f~\text{is holomorphic} \}. \]
\end{definition}

\begin{proposition}
  let $f, g \in \mathcal H(U)$.
  Then
  $f+g$ and $fg$ are in $\mathbb H(U)$ with $(fg)' = f'g + fg'$,
  and also $f/g$ with $(f/g)' = (f'g - fg') / g^2$ if $g\ne 0$.
  Moreover, if for $z \in U$ we have $f(z) \in U$, then
  \[(g \circ f)'(z) = g'(f(z)) f'(z).\]
\end{proposition}

\begin{definition}
  The \textbf{Cauchy-Riemann equations} are
  \[
  \frac{\partial u}{\partial x} = \frac{\partial v}{\partial y}
  \qquad\text{and}\qquad
  \frac{\partial u}{\partial y} = -\frac{\partial v}{\partial x}.
  \]
\end{definition}

\begin{proposition}
  If $f$ is holomorphic, then
  \[
  \frac{\partial f}{\partial \bar z} = 0
  \qquad\text{and}\qquad
  f'(z_0) = \frac{\partial f}{\partial z} = 2\frac{\partial u}{\partial z} (z_0).
  \]
\end{proposition}

\begin{theorem}
  If $u$ and $v$ are continuously differentiable and satisfy the Cauchy-Riemann equations,
  then $f$ is holomorphic.
\end{theorem}

\begin{definition}
  Let $z_0, a_n \in \mathbb C$ for all $n\in\mathbb N$.
  The formal infinite sum
  \[
  f(z) = \sum_{n=0}^\infty a_n (z - z_0)^n
  \]
  is called \textbf{power series around $z_0$}.
\end{definition}

\begin{theorem}
  Let $f$ be a power series.
  Then there exists a non-negative $R \in \mathbb R \cup \{\infty\}$ such that
  \begin{enumerate}
    \item $f(z)$ converges absolutely for all $z \in D_R(z_0)$
    \item $f(z)$ diverges for all $z \notin \overline D_R(z_0)$.
  \end{enumerate}
  $R$ is called the \textbf{Radius of convergence} and is given by
  $1/R = \limsup_{n \to \infty} \sqrt[n]{|a_n|}$
  where $1/0 = \infty$ and $1/\infty = 0$.
\end{theorem}

\begin{theorem}
  Let $f$ be a power series around $z_0 \in \mathbb C$ with radius of convergence $R$.
  Then $f$ is holomorphic on $D_R(z_0)$.
  $f'$ is also a power series around $z_0$ and radius of convergence $R$.
\end{theorem}

\begin{definition}
  A function $f$ is called \textbf{analytic}
  if there exists a power series $p(z)$ around $z_0 \in U$ with strictly positive radius of convergence
  such that $\forall z \in D_(z_0)\colon f(z) = p(z)$.
\end{definition}

\section{Integration along curves}

\end{document}

